Malwares are often analyzed in virtual machines (a.k.a. sandbox) for efficiency. However, sophisticated malwares are able to detect whether it is being executed on virtual machines or real hardware. If it is not on bare metal, then it holds back its malicious activities to hinder the analysis. In CCS\textquotesingle 08, Ether \cite{ether} has introduced a transparent malware analysis method. Three years lalter, P{\'e}k \textit{et al.}\cite{nether} published that Ether\cite{ether} was still detectable by using timing information, CPUID, and CPU errata. Although the fight between malware authors and security researchers on detecting and hiding virtual environment is an ongoing battle, achieving transparency is fundamentally more difficult~\cite{garfinkel2007}.
There are two types of virtualization in general. The first is software based where the guest OS runs on top of a host OS like the VMware or VirtualBox. The second is hardware based virtualization that runs without any host OS but on top of a hypervisor like Intel VT-x. The software based solution is far from perfect. It has countless bugs that it is infeasible to expect transparency. Although the hardware solution is much more stable, clever techniques such as testing the CPU cache remains to be a challenge for the security community.
In this paper we explore how nEther\cite{nether} has broken Ether\cite{ether} and potential mitigation methods for Ether. Our contributions are:

\begin{itemize}
\item We have surveyed Ether\cite{ether} detection techniques used by malware authors
\item We have proposed two novel mitigation method on Ether detection through CPU errata
\item We have emphasized analysis of sophisticated VMs should never be done on software based VM due to their numerous defects 
\end{itemize}

%%% Local Variables:
%%% mode: latex
%%% TeX-master: "../paper"
%%% End:
