Malware samples are often analyzed in virtual machines (a.k.a. sandbox
environments) to better understand how they work and, in the future, how they
can be defended against. However, sophisticated malware samples are able to
detect whether it is being executed on virtual machines or real hardware, as a
kind of anti-reverse engineering technique. If it is not run on bare metal, then
it holds back its malicious activities to thwart analysis. In
CCS\textquotesingle 08, Ether~\cite{ether} introduced a transparent malware
analysis method. Three years later, P{\'e}k \textit{et al.}~\cite{nether}
published that Ether~\cite{ether} was still detectable by using timing
information, CPUID, and CPU errata. Although the fight between malware authors
and security researchers on detecting and hiding virtual environment is an
ongoing battle, achieving transparency is fundamentally more
difficult~\cite{garfinkel2007}. There are two types of virtualization, in
general. The first is software-based, where the guest OS runs on top of a host
OS like the VMware or VirtualBox. The second is hardware-based virtualization
that runs without any host OS but on top of a hypervisor like Intel VT-x. The
software based solution is far from perfect. Perfect virtualization of hardware
in software is so difficult, that it full transparency is infeasible. Although
the hardware solution is more stable, techniques such as testing the CPU cache
remains a challenge for the security community. In this paper we explore how
nEther~\cite{nether} broke Ether's~\cite{ether} transparency and potential
mitigation methods for Ether. Our contributions are:

\begin{itemize}
\item We have surveyed Ether~\cite{ether} detection techniques used by malware
  authors
\item We have proposed two novel mitigation methods on Ether detection through
  CPU errata
\item We have emphasized how analysis software-based VMs are ineffective due to
  their numerous defects
\end{itemize}

%%% Local Variables:
%%% mode: latex
%%% TeX-master: "../paper"
%%% End:
